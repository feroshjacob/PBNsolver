\documentclass[11pt,stdletter,dateleft,orderfromtodate]{newlfm}
%\documentclass[11pt,stdletter]{newlfm}
\usepackage{charter}
\usepackage{url}

\widowpenalty=1000
\clubpenalty=1000
%\newlfmP{headermarginskip=1pt}
\newlfmP{sigsize=11pt}
%\newlfmP{dateskipafter=30pt}
%\newlfmP{addrfromphone}
%\newlfmP{addrfromemail}
%\PhrPhone{Phone}
%\PhrEmail{Email}

%\lthUiuc

\newcommand\addone{Professor Edward. K. Blum, Managing Editor }
\newcommand\addtwo{Journal of Computer and System Sciences }
\newcommand\addthree{Mathematics Department, University of Southern California, LA, CA-90089}
\newcommand\salute{Sir}

\namefrom{Ferosh Jacob \\
	112 Cumberland Gate Ln
	Smyrna, GA 30080\\
	Email: fjacob@crimson.ua.edu
}

\addrto{%
\addone \\ 
\addtwo \\ 
\addthree \\
}

\PhrRegard{Subject}
\regarding{\textbf{Journal of Computer and System Sciences  submission}}
\greetto{Dear \salute}
\closeline{Sincerely,}
\begin{document}
\begin{newlfm}

 I am writing to submit our paper titled ``PNBsolver: A Domain-Specific Language for Modeling Parallel 
 Nbody Problems`` for publication in the Journal of Computer and System Sciences.

 In this paper, we introduce a two-stage modeling approach that allows domain users to express the
 problem using domain constructs and reuse the available optimized solutions. This approach has been
 applied successfully to N-body problems using a domain-specific language called PNBsolver, which
 allows domain users to specify the computations in an N-body problem without any implementation or
 platform-specific details. Using the PNBsolver, the domain users are allowed to control the platform and
 implementation of the generated code. The accuracy and execution time of the generated code can also
 be fine-tuned based on the parameters provided in PNBsolver. We have included in this paper, some of
 the common N-body interactions showing how it can be implemented using PNBsolver.

 We confirm that this manuscript has not been published elsewhere and is not under consideration by
 another journal. All authors have approved the manuscript and agree with its submission to Journal of
 Computer and System Sciences.
\end{newlfm}
\end{document}

